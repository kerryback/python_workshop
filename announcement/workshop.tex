\documentclass[11pt]{scrartcl}
\usepackage{graphicx}
\usepackage{hyperref}
\usepackage{geometry}
\usepackage{enumitem}

\geometry{margin=1.25in, top=1in}

\begin{document}

\begin{center}
    \includegraphics[width=0.4\textwidth]{RiceBusiness-transparent-logo-sm.png}
\end{center}

\begin{center}
\large \textbf{MBA Python Workshop\\ September 6, 2025}

\end{center}

\vskip \baselineskip

Python has become the world's most popular programming language. From Fortune 500 companies to startups, organizations are leveraging python to drive data-driven decision making and to automate complex processes. Its intuitive syntax and extensive ecosystem of business-focused libraries make it an ideal tool for business professionals.

Due to the emergence of generative AI, it is no longer necessary to spend years of study or to memorize syntax to use python effectively.  What business professionals need is an understanding of the overall structure of the language and some familiarity with the key libraries for data analysis and business applications.  The details that would have been stumbling blocks for effective use a few years ago can now be handled by AI.  

This free non-credit one-day workshop is designed to take students from zero awareness of how programming works and zero knowledge of python to the point of being able to use python effectively with AI assistance.  It is a hands-on workshop.  No software installation is needed.  We will work entirely in Google Colab, which is a free python environment in the cloud with built-in assistance from Google Gemini.  Learning objectives include:

\begin{itemize}[itemsep=0ex]
\item How to navigate a Jupyter notebook
\item How to get data into Google Colab and how to get results out
\item Python data types (floats, integers, strings, booleans)
\item Key python objects (lists, dictionaries)
\item Flow control in python (loops, if-else)
\item How to create and use python functions
\item Methods and attributes of python objects (what is all that abc.xyz stuff about?)
\item How to import python libraries
\item How to work with data tables with the pandas library
\item How to do statistics with the pandas and statsmodels libraries
\item How to visualize data with the matplotlib and seaborn libraries
\item How to do simulation with the numpy library
\item How to do goal-seek with the scipy library (if time is available)
\item How to do build apps with the streamlit library (if time is available)
\end{itemize}

The workshop is scheduled to run from 9:00 am to 3:00 pm.  Lunch will be provided.  The workshop will be directed by Kerry Back, J. Howard Creekmore Professor of Finance and Professor of Economics, Rice University.    JGSB PhD students will assist in providing individual attention as needed throughout the workshop.
\end{document}
